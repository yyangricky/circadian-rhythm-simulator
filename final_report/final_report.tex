\documentclass{article}
\usepackage{amsmath}
\usepackage{parskip}

\title{Modeling Recovery after Circadian Rhythm Disruption using Differential Equation Models}
\author{Lara Simpson, Mark Saad, Ricky Yang, Chris Demian}
\date{\today}

\begin{document}

\maketitle
\newpage
\section{Abstract}
We investigated circadian rhythm recovery dynamics by creating sleep disruption and comparing four models, all consisting of differential equations: Forger [1], Jewett[2], and Hannay [3](both singe and two-population). We simulated disruption scenarios of three weeks of delayed weekend sleep schedules (social jet lag). Solutions were computed using 4th-order Runge-Kutta integration with CBTmin as the phase marker. All models achieved re-entrainment within 3 days under light-dark cycles for social jet lag. In contrast, when in darkness, models exhibited free-running behaviour with drift rates of 10.8-15.9 min/day, confirming that zeitgeber cues are necessary for entrainment. These findings confirm the robustness of circadian phase predictions across different models.
\section{Introduction}
Modelling human circadian rhythm is a topic of great interest in biology as it is the biological mechanism that governs the periodical processes of the body, such as sleep-wake cycles. Circadian rhythm is almost exclusively modelled by ODEs, making this topic highly suitable for this course. The topic of circadian rhythms is of personal relevance to the authors, as we all share the universal university experience of sleep loss before a midterm, which leads us to wonder if we can mathematically model what happens to our circadian rhythm after repeated late nights and all-nighters. This leads to the research question: how long does it take for the circadian rhythm to recover after being disrupted? We will be comparing the models in the papers of Forger, Jewett, and Hannay.
\section{Mathmatical Model and Theoretical Background}
All models of circadian rhythms utilizes the Van der Pol’s oscillator. The basic model that the more recent models are based on was first proposed by Richard Kronauer in 1990 [4] and consists of the following pair of coupled ODES:
\begin{equation}
    \frac{dx}{dt} = \left( \frac{\pi}{12} \right) \left[ x_c + \mu \left( x - \frac{4x^3}{3} \right) + B \right]
\end{equation}

\begin{equation}
    \frac{dx_c}{dt} = \left( \frac{\pi}{12} \right) \left[ -\left( \frac{24}{\tau_x} \right)^2 x + Bx_c \right]
\end{equation}
Here, x is the primary circadian oscillator indicating the position along the circadian cycle (i.e. wakefuness, body temperature, etc) . $x_c$ is the auxiliary variable maintaining stability in the system. $\tau$ is the intrinsic period of 24.2h, and light enters the system as the perceived brightness B. (We will not be implimenting the Kronauer model)
An adaptation to that model is presented by Forger et. al in 1999 [1] which attempts to model it more accurately and simply. The cubic term is moved from the $\frac{dx}{dt}$ equation to the auxiliary equation as the amplitude of the limit cycle is more dependent on the strength of the light drive, and a small correction factor of $\left(\frac{1}{0.99669}\right)^2$ is added. A $kBx$ term is also added to account for the effect of light on the circadian rhythm in the spirit of Aschoff’s rule. Note that light intensity is converted to $B$ through a process called Process L, consisting of the equations (5) - (9).
\begin{equation}
    \frac{dx}{dt} = \frac{\pi}{12}(x_c + B)
\end{equation}

\begin{equation}
    \frac{dx_c}{dt} = \frac{\pi}{12}\left\{\mu\left(x_c - \frac{4x_c^3}{3}\right) - x\left[\left(\frac{24}{0.99669\tau_x}\right)^2 + kB\right]\right\}
\end{equation}

\begin{equation}
\alpha(I) = \alpha_0\left(\frac{I^p}{I_0^p}\right)
\end{equation}


\begin{equation}
\frac{dn}{dt} = 60[\alpha(I)(1-n) - \beta n]
\end{equation}

\begin{equation}
\hat{B} = G(1-n)\alpha(I)
\end{equation}


\begin{equation}
B = \hat{B}(1-0.4x)(1-0.4x_c)
\end{equation}


\begin{equation}
\text{CBT}_{\min} = x_{\min} + \phi_{ref}
\end{equation}
\text{Parameters:} $\alpha_0 = 0.16$, $\beta = 0.013$, $G = 19.875$, $p = 0.6$, $k = 0.55$, $I_0 = 9500$ lux.
where $\phi_{ref} = 0.97$ hours is a phase reference correction.


Another adaptation to Kronauer’s model is presented by Jewett et al in 1999. Again, light intensity is converted to B through Process L
\begin{equation}
    \dot{x} = \left(\frac{\pi}{12}\right)\left[x_c + \mu\left(\frac{1}{3}x + \frac{4}{3}x^3 - \frac{256}{105}x^7\right) + B\right]
    \end{equation}
\begin{equation}
        \dot{x}_c = \left(\frac{\pi}{12}\right)\left\{qBx_c - \left[\left(\frac{24}{0.99729\tau_x}\right)^2 + kB\right]x\right\}
    \end{equation}
More recently, in 2019, Hannay et al presented two much more complex models that introduce a macroscopic approach derived from SCN neuron network models. Hannay’s “single-population” model uses an explicit amplitude state R (radius of oscillation) as one of the state variables and a phase angle (or an equivalent two-state formulation where state[0] behaves like an amplitude variable). It is a self-sustained oscillator with parameters tuned to human data, and it can exhibit more realistic behaviour under strong perturbations (e.g., slower amplitude recovery). Meanwhile, the model with two populations (often representing two subdivisions of the SCN) This model allows for internal desynchrony between two coupled oscillators, which can capture complex dynamics like transients in amplitude and phase that single-oscillator models might miss. It’s an advanced model expected to handle large perturbations (like all-nighters) more realistically by, for example, showing significant amplitude suppression and slow recovery when cues are removed. 


\textbf{Single Population:}
\begin{equation}
\dot{R} = -(D+\gamma)R + \frac{K}{2}\cos(\beta)R(1-R^4) + L_R(R,\psi)
\end{equation}

\begin{equation}
\dot{\psi} = \omega_0 + \frac{K}{2}\sin(\beta)(1+R^4) + L_\psi(R,\psi)
\end{equation}

\begin{equation}
L_R(R,\psi) = \frac{A_1}{2}B(t)(1-R^4)\cos(\psi + \beta_{L1}) + \frac{A_2}{2}B(t)R(1-R^8)\cos(2\psi + \beta_{L2})
\end{equation}

\begin{equation}
L_\psi(R,\psi) = \sigma B(t) - \frac{A_1}{2}B(t)\left(\frac{1}{R} + R^3\right)\sin(\psi + \beta_{L1}) - \frac{A_2}{2}B(t)(1+R^8)\sin(2\psi + \beta_{L2})
\end{equation}

\textbf{Two Population:}
\begin{equation}
\dot{R}_v = -\gamma R_v + \frac{K_{vv}}{2}R_v(1-R_v^4) + \frac{K_{dv}}{2}R_d(1-R_v^4)\cos(\psi_d - \psi_v) + L_R(R_v,\psi_v)
\end{equation}

\begin{equation}
\dot{R}_d = -\gamma R_d + \frac{K_{dd}}{2}R_d(1-R_d^4) + \frac{K_{vd}}{2}R_v(1-R_d^4)\cos(\psi_d - \psi_v)
\end{equation}

\begin{equation}
\dot{\psi}_v = \omega_v + \frac{K_{dv}}{2}R_d\left(\frac{1}{R_v} + R_v^3\right)\sin(\psi_d - \psi_v) + L_\psi(R_v,\psi_v)
\end{equation}

\begin{equation}
\dot{\psi}_d = \omega_d - \frac{K_{vd}}{2}R_v\left(\frac{1}{R_d} + R_d^3\right)\sin(\psi_d - \psi_v)
\end{equation}

\begin{equation}
L_R = \frac{A_1}{2}B(t)(1-R_v^4)\cos(\psi_v + \beta_{L1}) + \frac{A_2}{2}B(t)R_v(1-R_v^8)\cos(2\psi_v + \beta_{L2})
\end{equation}

\begin{equation}
L_\psi = \sigma B(t) - \frac{A_1}{2}B(t)\left(\frac{1}{R_v} + R_v^3\right)\sin(\psi_v + \beta_{L1}) - \frac{A_2}{2}B(t)(1+R_v^8)\sin(2\psi_v + \beta_{L2})
\end{equation}

Note that $K_{from,to}$ represents coupling strength from one region to another.

\section{Methodology}
We set up a program modelling circadian rhythm, implementing the four models mentioned above. Solutions are computed using the 4th-order Runge-Kutta method. The input is the light level as a function of time, I(t). Core body temperature minimum (CBTmin), which occurs in early morning hours, is a standard physiological marker used as the state variable, x.  We set the light input to 0 when the person goes to bed, and 250 lux when the person is up.

To simulate disruptions to a normal schedule, we assume a person who wakes up at 7:00 and goes to sleep at 23:00 under normal circumstances. We first run the program under a normal light schedule for the circadian rhythm to reach an equilibrium, then we simulate disruptions to their circadian rhythm: The light schedule is normal the first 30 days for x to equilibrate. Then, for the next 3 weeks, the person stays up until 2:00 and wakes up at 10:00 (3 hour delay) for Saturday and Sunday only. The light schedule returns to normal after day 49 and runs for another 3 weeks for the circadian rhythm to re-equilibrate. Then, the simulation is repeated, except we force complete darkness after the disruption to calculate darkness recovery.

The baseline CBTmin is computed by averaging the clock hour of CBTmin across (up to) the first 4 weeks. To compute re-entrainment time, the program checks when the model’s daily CBTmin is within 15 minutes of the baseline phase (relative phase error) for three consecutive days. The streak of 3 days corresponds to common practice in jet lag studies. If reached, the function returns the time (in hours) from the start of recovery to the first day of that streak. This is reported for both darkness and LD conditions. 90\% amplitude recovery time is also computed for both light and dark conditions.
\section{Results}
The actogram (Figure 1) displays CBTmin timing across days on the y-axis and zeitgeber time on the x-axis, with light/dark schedule bars showing light exposure. During baseline, all four models maintained tight oscillating patterns. This shows that initially there was stable entrainment with CBTmin occurring consistently at 4:00-4:45 AM. The disruption phase produced progressive widening and drift of these oscillating patterns, showing accumulating phase delay from repeated weekend delays. Recovery shows the oscillations compressing and returning back to baseline as the models re-entrain to regular schedules.

Phase Deviation analysis (Figure 2) tracked deviations from baseline across the recovery period for LD cycles versus conditions of constant darkness. All four models accumulated peak deviations of 73-88 minutes by the end of the disruption period. Under LD recover, all models achieved re-entrainment which is defined as three consecutive days within ±20 minutes of baseline, at exactly 72 hours. In contrast, constant darkness conditions did not have zeitgeber input and displayed persistent free-running behaviour with linear drift rates of 10.8-15.9 min/day. These drift rates correspond to intrinsic circadian periods of approximately 24.2 hours. This is consistent with human isolation studies and confirms that light zeitgebers are necessary for stable entrainment. 

Our quantitative results are summarized in the table below.
\section{Discussion}
Interpret your findings, identify limitations, and suggest further work.
\section{Conclusion}
Summarize your contribution.
\section{References}
Include a properly formatted bibliography.
\section{Appendix}
For appendix, include rough code that did not make it into the report

\end{document}
