\documentclass{article}

\title{Modeling Recovery after Circadian Rhythm Disruption}
\author{Lara Simpson, Mark Saad, Ricky Yang, Chris Demian}
\date{\today}

\begin{document}

\maketitle
\newpage
\section{Abstract}
A brief summary of the objectives, methods, and key findings
\section{Introduction}
Modelling human circadian rhythm is a topic of great interest in biology as it is the
biological mechanism that governs the periodical processes of the body, such as sleep-wake cycles. Circadian rhythm is almost exclusively modelled by
ODEs, making this topic highly suitable for this course. The topic of circadian rhythms is of personal relevance
to the authors as we all share the universal university experience of sleep loss before a midterm, which leads us to wonder if we can mathmatecally model what happens to our circadian rhythm after we pull an all-nighter. This leads to the research question: how long it takes for the circadian rhythm to recover after being disrupted by an all nighter? Due to the existance of multiple established models of circadian rhythm we will be comparing the models in the papers of Forger, Jewett, and Hannay and evaluating their accurcies using the Recovery Ratio. 
\section{Mathmatical Model and Theoretical Background}
Present core equations, derivations, or theoretical tools.
\section{Methodology}
Detail any computational/numerical methods or experimental setup.
\section{Results}
Present and interpret key results using figures, graphs, or tables.
\section{Discussion}
Interpret your findings, identify limitations, and suggest further work.
\section{Conclusion}
Summarize your contribution.
\section{References}
Include a properly formatted bibliography.
\section{Appendix}
For appendix, include rough code that did not make it into the report

\end{document}